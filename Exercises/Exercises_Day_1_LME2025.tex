% Options for packages loaded elsewhere
\PassOptionsToPackage{unicode}{hyperref}
\PassOptionsToPackage{hyphens}{url}
%
\documentclass[
]{article}
\usepackage{amsmath,amssymb}
\usepackage{iftex}
\ifPDFTeX
  \usepackage[T1]{fontenc}
  \usepackage[utf8]{inputenc}
  \usepackage{textcomp} % provide euro and other symbols
\else % if luatex or xetex
  \usepackage{unicode-math} % this also loads fontspec
  \defaultfontfeatures{Scale=MatchLowercase}
  \defaultfontfeatures[\rmfamily]{Ligatures=TeX,Scale=1}
\fi
\usepackage{lmodern}
\ifPDFTeX\else
  % xetex/luatex font selection
\fi
% Use upquote if available, for straight quotes in verbatim environments
\IfFileExists{upquote.sty}{\usepackage{upquote}}{}
\IfFileExists{microtype.sty}{% use microtype if available
  \usepackage[]{microtype}
  \UseMicrotypeSet[protrusion]{basicmath} % disable protrusion for tt fonts
}{}
\makeatletter
\@ifundefined{KOMAClassName}{% if non-KOMA class
  \IfFileExists{parskip.sty}{%
    \usepackage{parskip}
  }{% else
    \setlength{\parindent}{0pt}
    \setlength{\parskip}{6pt plus 2pt minus 1pt}}
}{% if KOMA class
  \KOMAoptions{parskip=half}}
\makeatother
\usepackage{xcolor}
\usepackage[margin=1in]{geometry}
\usepackage{color}
\usepackage{fancyvrb}
\newcommand{\VerbBar}{|}
\newcommand{\VERB}{\Verb[commandchars=\\\{\}]}
\DefineVerbatimEnvironment{Highlighting}{Verbatim}{commandchars=\\\{\}}
% Add ',fontsize=\small' for more characters per line
\usepackage{framed}
\definecolor{shadecolor}{RGB}{248,248,248}
\newenvironment{Shaded}{\begin{snugshade}}{\end{snugshade}}
\newcommand{\AlertTok}[1]{\textcolor[rgb]{0.94,0.16,0.16}{#1}}
\newcommand{\AnnotationTok}[1]{\textcolor[rgb]{0.56,0.35,0.01}{\textbf{\textit{#1}}}}
\newcommand{\AttributeTok}[1]{\textcolor[rgb]{0.13,0.29,0.53}{#1}}
\newcommand{\BaseNTok}[1]{\textcolor[rgb]{0.00,0.00,0.81}{#1}}
\newcommand{\BuiltInTok}[1]{#1}
\newcommand{\CharTok}[1]{\textcolor[rgb]{0.31,0.60,0.02}{#1}}
\newcommand{\CommentTok}[1]{\textcolor[rgb]{0.56,0.35,0.01}{\textit{#1}}}
\newcommand{\CommentVarTok}[1]{\textcolor[rgb]{0.56,0.35,0.01}{\textbf{\textit{#1}}}}
\newcommand{\ConstantTok}[1]{\textcolor[rgb]{0.56,0.35,0.01}{#1}}
\newcommand{\ControlFlowTok}[1]{\textcolor[rgb]{0.13,0.29,0.53}{\textbf{#1}}}
\newcommand{\DataTypeTok}[1]{\textcolor[rgb]{0.13,0.29,0.53}{#1}}
\newcommand{\DecValTok}[1]{\textcolor[rgb]{0.00,0.00,0.81}{#1}}
\newcommand{\DocumentationTok}[1]{\textcolor[rgb]{0.56,0.35,0.01}{\textbf{\textit{#1}}}}
\newcommand{\ErrorTok}[1]{\textcolor[rgb]{0.64,0.00,0.00}{\textbf{#1}}}
\newcommand{\ExtensionTok}[1]{#1}
\newcommand{\FloatTok}[1]{\textcolor[rgb]{0.00,0.00,0.81}{#1}}
\newcommand{\FunctionTok}[1]{\textcolor[rgb]{0.13,0.29,0.53}{\textbf{#1}}}
\newcommand{\ImportTok}[1]{#1}
\newcommand{\InformationTok}[1]{\textcolor[rgb]{0.56,0.35,0.01}{\textbf{\textit{#1}}}}
\newcommand{\KeywordTok}[1]{\textcolor[rgb]{0.13,0.29,0.53}{\textbf{#1}}}
\newcommand{\NormalTok}[1]{#1}
\newcommand{\OperatorTok}[1]{\textcolor[rgb]{0.81,0.36,0.00}{\textbf{#1}}}
\newcommand{\OtherTok}[1]{\textcolor[rgb]{0.56,0.35,0.01}{#1}}
\newcommand{\PreprocessorTok}[1]{\textcolor[rgb]{0.56,0.35,0.01}{\textit{#1}}}
\newcommand{\RegionMarkerTok}[1]{#1}
\newcommand{\SpecialCharTok}[1]{\textcolor[rgb]{0.81,0.36,0.00}{\textbf{#1}}}
\newcommand{\SpecialStringTok}[1]{\textcolor[rgb]{0.31,0.60,0.02}{#1}}
\newcommand{\StringTok}[1]{\textcolor[rgb]{0.31,0.60,0.02}{#1}}
\newcommand{\VariableTok}[1]{\textcolor[rgb]{0.00,0.00,0.00}{#1}}
\newcommand{\VerbatimStringTok}[1]{\textcolor[rgb]{0.31,0.60,0.02}{#1}}
\newcommand{\WarningTok}[1]{\textcolor[rgb]{0.56,0.35,0.01}{\textbf{\textit{#1}}}}
\usepackage{graphicx}
\makeatletter
\def\maxwidth{\ifdim\Gin@nat@width>\linewidth\linewidth\else\Gin@nat@width\fi}
\def\maxheight{\ifdim\Gin@nat@height>\textheight\textheight\else\Gin@nat@height\fi}
\makeatother
% Scale images if necessary, so that they will not overflow the page
% margins by default, and it is still possible to overwrite the defaults
% using explicit options in \includegraphics[width, height, ...]{}
\setkeys{Gin}{width=\maxwidth,height=\maxheight,keepaspectratio}
% Set default figure placement to htbp
\makeatletter
\def\fps@figure{htbp}
\makeatother
\setlength{\emergencystretch}{3em} % prevent overfull lines
\providecommand{\tightlist}{%
  \setlength{\itemsep}{0pt}\setlength{\parskip}{0pt}}
\setcounter{secnumdepth}{-\maxdimen} % remove section numbering
\ifLuaTeX
  \usepackage{selnolig}  % disable illegal ligatures
\fi
\IfFileExists{bookmark.sty}{\usepackage{bookmark}}{\usepackage{hyperref}}
\IfFileExists{xurl.sty}{\usepackage{xurl}}{} % add URL line breaks if available
\urlstyle{same}
\hypersetup{
  pdftitle={Exercises Day 1},
  pdfauthor={Hannah S. Heinrichs},
  hidelinks,
  pdfcreator={LaTeX via pandoc}}

\title{Exercises Day 1}
\author{Hannah S. Heinrichs}
\date{2025-06-02}

\begin{document}
\maketitle

\hypertarget{exercise-1}{%
\section{Exercise 1}\label{exercise-1}}

Suppose you are running a behavioural experiment, studying the mental
speed of young and old people. You know that young people have an
average reaction time of 550 with a standard deviation of 120.

\begin{enumerate}
\def\labelenumi{\alph{enumi})}
\tightlist
\item
  What is the probability of getting a reaction time of 420 or lower
  from a single young person?
\end{enumerate}

\begin{Shaded}
\begin{Highlighting}[]
\NormalTok{prob }\OtherTok{\textless{}{-}} \FunctionTok{pnorm}\NormalTok{(}\DecValTok{420}\NormalTok{,}\AttributeTok{mean=}\DecValTok{550}\NormalTok{,}\AttributeTok{sd=}\DecValTok{120}\NormalTok{, }\AttributeTok{lower.tail =} \ConstantTok{TRUE}\NormalTok{)}
\FunctionTok{round}\NormalTok{(prob }\SpecialCharTok{*} \DecValTok{100}\NormalTok{, }\DecValTok{2}\NormalTok{)}
\end{Highlighting}
\end{Shaded}

\begin{verbatim}
## [1] 13.93
\end{verbatim}

\begin{enumerate}
\def\labelenumi{\alph{enumi})}
\setcounter{enumi}{1}
\tightlist
\item
  What is the probability of getting a reaction time of 560 or higher
  from a single young person?
\end{enumerate}

\begin{Shaded}
\begin{Highlighting}[]
\NormalTok{prob }\OtherTok{\textless{}{-}} \FunctionTok{pnorm}\NormalTok{(}\DecValTok{560}\NormalTok{,}\AttributeTok{mean=}\DecValTok{550}\NormalTok{,}\AttributeTok{sd=}\DecValTok{120}\NormalTok{)}
\FunctionTok{round}\NormalTok{(prob }\SpecialCharTok{*} \DecValTok{100}\NormalTok{, }\DecValTok{2}\NormalTok{)}
\end{Highlighting}
\end{Shaded}

\begin{verbatim}
## [1] 53.32
\end{verbatim}

\begin{enumerate}
\def\labelenumi{\alph{enumi})}
\setcounter{enumi}{2}
\tightlist
\item
  For your study, you need 10 reaction time values for young people. How
  can you generate a sample with n = 10?
\end{enumerate}

\begin{Shaded}
\begin{Highlighting}[]
\NormalTok{sample }\OtherTok{\textless{}{-}} \FunctionTok{rnorm}\NormalTok{(}\DecValTok{10}\NormalTok{, }\AttributeTok{mean=}\DecValTok{550}\NormalTok{, }\AttributeTok{sd=}\DecValTok{120}\NormalTok{)}
\NormalTok{sample}
\end{Highlighting}
\end{Shaded}

\begin{verbatim}
##  [1] 640.8042 653.3911 554.0180 657.6380 664.2018 513.8322 362.4063 588.6874
##  [9] 671.9529 127.0435
\end{verbatim}

\begin{enumerate}
\def\labelenumi{\alph{enumi})}
\setcounter{enumi}{3}
\tightlist
\item
  You get 15 values from a population of old people. What is the mean
  and the standart deviation of their underlying normal distribution?
\end{enumerate}

\begin{Shaded}
\begin{Highlighting}[]
\NormalTok{x }\OtherTok{\textless{}{-}} \FunctionTok{c}\NormalTok{(}\FloatTok{434.7}\NormalTok{, }\FloatTok{671.4}\NormalTok{, }\FloatTok{428.9}\NormalTok{, }\FloatTok{454.4}\NormalTok{, }\FloatTok{806.1}\NormalTok{, }\FloatTok{483.3}\NormalTok{, }\FloatTok{819.1}\NormalTok{, }\FloatTok{630.4}\NormalTok{, }\FloatTok{836.2}\NormalTok{, }\FloatTok{661.4}\NormalTok{, }\FloatTok{511.7}\NormalTok{, }\FloatTok{507.2}\NormalTok{, }\FloatTok{568.0}\NormalTok{, }\FloatTok{707.9}\NormalTok{, }\FloatTok{621.7}\NormalTok{)}

\NormalTok{mean\_x }\OtherTok{\textless{}{-}} \FunctionTok{mean}\NormalTok{(x)}
\NormalTok{sd\_x }\OtherTok{\textless{}{-}} \FunctionTok{sd}\NormalTok{(x)}
\NormalTok{out }\OtherTok{\textless{}{-}} \FunctionTok{paste0}\NormalTok{(}\StringTok{"$"}\NormalTok{, mean\_x, }\StringTok{" }\SpecialCharTok{\textbackslash{}\textbackslash{}}\StringTok{pm "}\NormalTok{, sd\_x, }\StringTok{"$"}\NormalTok{)}
\end{Highlighting}
\end{Shaded}

\begin{enumerate}
\def\labelenumi{\alph{enumi})}
\setcounter{enumi}{4}
\tightlist
\item
  What is the probability that the 15 values are actually drawn from the
  same distribution as in 1c
\end{enumerate}

\begin{Shaded}
\begin{Highlighting}[]
\NormalTok{tstat }\OtherTok{\textless{}{-}} \FunctionTok{t.test}\NormalTok{(sample, x, }\AttributeTok{paired =} \ConstantTok{FALSE}\NormalTok{)}
\FunctionTok{round}\NormalTok{(tstat}\SpecialCharTok{$}\NormalTok{p.value}\SpecialCharTok{*}\DecValTok{100}\NormalTok{, }\DecValTok{2}\NormalTok{)}
\end{Highlighting}
\end{Shaded}

\begin{verbatim}
## [1] 33.17
\end{verbatim}

\hypertarget{exercise-2}{%
\section{Exercise 2}\label{exercise-2}}

You are a researcher studying the intelligence of dragons in the
european mountains. You suspect that larger dragons are also smarter and
you've collected various samples for intelligence (testscore) and size
(bodylength) from different mountains.

\begin{enumerate}
\def\labelenumi{\alph{enumi})}
\tightlist
\item
  Is there a significant relationship between intelligence and body size
  in dragons? Use linear regression.
\end{enumerate}

\begin{Shaded}
\begin{Highlighting}[]
\FunctionTok{load}\NormalTok{(}\StringTok{"dragons.RData"}\NormalTok{)}

\CommentTok{\# Normal distribution}


\CommentTok{\# Test correlation}
\NormalTok{model }\OtherTok{\textless{}{-}} \FunctionTok{cor.test}\NormalTok{(dragons}\SpecialCharTok{$}\NormalTok{testScore, dragons}\SpecialCharTok{$}\NormalTok{bodyLength, }\AttributeTok{method=}\StringTok{"pearson"}\NormalTok{)}
\NormalTok{model}\SpecialCharTok{$}\NormalTok{p.value }\SpecialCharTok{\textless{}} \FloatTok{0.05}
\end{Highlighting}
\end{Shaded}

\begin{verbatim}
## [1] TRUE
\end{verbatim}

\begin{enumerate}
\def\labelenumi{\alph{enumi})}
\setcounter{enumi}{1}
\tightlist
\item
  Maybe the location of each recording sample influences the results.
  Create a LME that accounts for differences between mountain ranges.
  What is the relationship between intelligence and body size now?
\end{enumerate}

\begin{Shaded}
\begin{Highlighting}[]
\FunctionTok{head}\NormalTok{(dragons)}
\end{Highlighting}
\end{Shaded}

\begin{verbatim}
## # A tibble: 6 x 6
## # Rowwise: 
##   testScore bodyLength mountainRange site  normal_mean Color
##       <dbl>      <dbl> <fct>         <fct>       <int> <chr>
## 1      16.1       166. Bavarian      a               1 Green
## 2      44.6       168. Bavarian      a               0 Red  
## 3      24.9       166. Bavarian      a               0 Red  
## 4      18.8       168. Bavarian      a               1 Green
## 5      33.9       170. Bavarian      a               1 Green
## 6      47.0       169. Bavarian      a               1 Green
\end{verbatim}

\begin{Shaded}
\begin{Highlighting}[]
\NormalTok{formula }\OtherTok{\textless{}{-}} \StringTok{"testScore \textasciitilde{} bodyLength + (1 | mountainRange)"}
\NormalTok{model }\OtherTok{\textless{}{-}}\NormalTok{ lme4}\SpecialCharTok{::}\FunctionTok{lmer}\NormalTok{(formula, }\AttributeTok{data =}\NormalTok{ dragons)}
\FunctionTok{summary}\NormalTok{(model)}
\end{Highlighting}
\end{Shaded}

\begin{verbatim}
## Linear mixed model fit by REML ['lmerMod']
## Formula: testScore ~ bodyLength + (1 | mountainRange)
##    Data: dragons
## 
## REML criterion at convergence: 4187.3
## 
## Scaled residuals: 
##      Min       1Q   Median       3Q      Max 
## -2.69490 -0.68646 -0.01821  0.67354  2.64609 
## 
## Random effects:
##  Groups        Name        Variance Std.Dev.
##  mountainRange (Intercept) 291.2    17.07   
##  Residual                  340.1    18.44   
## Number of obs: 480, groups:  mountainRange, 8
## 
## Fixed effects:
##             Estimate Std. Error t value
## (Intercept) 31.06599   20.25713   1.534
## bodyLength   0.14447    0.09597   1.505
## 
## Correlation of Fixed Effects:
##            (Intr)
## bodyLength -0.954
\end{verbatim}

\begin{enumerate}
\def\labelenumi{\alph{enumi})}
\setcounter{enumi}{2}
\tightlist
\item
  You notice that dragons with different colors behave differently.
  Control in your LME for the color of each dragon. How do the results
  change?
\end{enumerate}

\begin{Shaded}
\begin{Highlighting}[]
\NormalTok{formula }\OtherTok{\textless{}{-}} \StringTok{"testScore \textasciitilde{} bodyLength + Color + (1 | mountainRange)"}
\NormalTok{model }\OtherTok{\textless{}{-}}\NormalTok{ lme4}\SpecialCharTok{::}\FunctionTok{lmer}\NormalTok{(formula, }\AttributeTok{data =}\NormalTok{ dragons) }
\FunctionTok{summary}\NormalTok{(model)}
\end{Highlighting}
\end{Shaded}

\begin{verbatim}
## Linear mixed model fit by REML ['lmerMod']
## Formula: testScore ~ bodyLength + Color + (1 | mountainRange)
##    Data: dragons
## 
## REML criterion at convergence: 4014.8
## 
## Scaled residuals: 
##     Min      1Q  Median      3Q     Max 
## -3.6979 -0.6275  0.0068  0.6859  2.6846 
## 
## Random effects:
##  Groups        Name        Variance Std.Dev.
##  mountainRange (Intercept) 337.4    18.37   
##  Residual                  237.1    15.40   
## Number of obs: 480, groups:  mountainRange, 8
## 
## Fixed effects:
##             Estimate Std. Error t value
## (Intercept) 34.80775   17.54353   1.984
## bodyLength   0.07632    0.08097   0.943
## ColorRed    20.38107    1.42452  14.307
## 
## Correlation of Fixed Effects:
##            (Intr) bdyLng
## bodyLength -0.927       
## ColorRed    0.005 -0.048
\end{verbatim}

Relationship much weaker, when correcting for Color, indicating that
Color is a confounding variable to the relationship between bodyLength
and testScore.

\end{document}
